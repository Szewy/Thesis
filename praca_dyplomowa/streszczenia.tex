\newpage
\begin{center}
\large \bf
Rozpoznawanie gestów na przykładzie gry papier, kamień, nożyce
\end{center}

\section*{Streszczenie}
W pracy inżynierskiej zostało przedstawione podejście do problemu rozpoznawania gestów przez wszystkie jego etapy. Na początku zostało pokazane zabranie i przetwarzanie danych w celu uzyskania jak najlepszego zbioru uczącego jak i testowego. Potem zostały opisane etapy filtracji, segmentacji dłoni. Po uzyskaniu zbioru danych zostały stworzone klasyfikatory. Pierwszym podejściem było stworzenie klasyfikatora KNN, a następnie stworzenie klasyfikatora bazującego na sieciach neuronowych. Końcowym etapy pracy to porównanie wyników uzyskanych przez wymienione klasyfikatory na przykładzie gry papier, kamień, nożyce.

\bigskip
{\noindent\bf Słowa kluczowe:} rozpoznawanie gestów, klasyfikacja, KNN, OpenCV, TensorFlow, sieci neuronowe

\vskip 2cm


\begin{center}
\large \bf
ENG Rozpoznawanie gestów na przykładzie gry papier, kamień, nożyce
\end{center}

\section*{Abstract}

\bigskip
{\noindent\bf Keywords:} ENG rozpoznawanie gestów, klasyfikacja, python, sieci neuronowe

\vfill