\newpage
\begin{center}
\large \bf
Rozpoznawanie gestów na przykładzie gry papier, kamień, nożyce
\end{center}

\section*{Streszczenie}
W pracy inżynierskiej zostało przedstawione podejście do problemu rozpoznawania gestów przez wszystkie jego etapy. W pierwszym etapie został przedstawiony zakres i cel pracy. Kolejne dwa rozdziały to teoretyczny wstęp do zagadnień oraz wykorzystanych narzędzi w pracy. 

Rozdział ,,rozwiązywanie problemu'' pokazuje podejście do klasyfikacji gestów na przykładzie gry papier, kamień, nożyce. Obrazuje aktywizację oraz  przetwarzanie danych w celu uzyskania jak najlepszego zbioru danych. W późniejszej części zostały opisane etapy filtracji, segmentacji dłoni. Po uzyskaniu zbioru danych zostały stworzone klasyfikatory. Pierwszym podejściem było stworzenie klasyfikatora KNN, a następnie stworzenie klasyfikatora bazującego na sieciach neuronowych. 

Zakończenie pracy to podsumowanie wykonanej pracy, porównanie wyników uzyskanych przez użyte klasyfikatory oraz opisanie wniosków zaobserwowanych podczas wykonywania pracy inżynierskiej.

\bigskip
{\noindent\bf Słowa kluczowe:} rozpoznawanie gestów, klasyfikacja, sieci neuronowe, OpenCV, TensorFlow, Keras, klasyfikator KNN

\vskip 2cm


\begin{center}
\large \bf
ENG Rozpoznawanie gestów na przykładzie gry papier, kamień, nożyce
\end{center}

\section*{Abstract}

\bigskip
{\noindent\bf Keywords:} ENG rozpoznawanie gestów, klasyfikacja, python, sieci neuronowe

\vfill